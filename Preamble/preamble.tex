% ******************************************************************************
% ****************************** Custom Margin *********************************

% Add `custommargin' in the document class options to use this section
% Set {innerside margin / outerside margin / topmargin / bottom margin}  and
% other page dimensions
\ifsetMargin
\else
    \RequirePackage[left=10mm,right=10mm,top=5mm,bottom=10mm]{geometry}
    \setFancyHdr % To apply fancy header after geometry package is loaded
\fi

% *****************************************************************************
% ******************* Fonts (like different typewriter fonts etc.)*************

% Add `customfont' in the document class option to use this section
\ifsetFont
\else
    % Set your custom font here and use `customfont' in options. Leave empty to
    % load computer modern font (default LaTeX font).  
    \RequirePackage{libertine} 
\fi

% *****************************************************************************
% *************************** Bibliography  and References ********************

%\usepackage{cleveref} %Referencing without need to explicitly state fig /table
%\usepackage[round]{natbib}
% Add `custombib' in the document class option to use this section
\ifsetBib % True, Bibliography option is chosen in class options
\else % If custom bibliography style chosen then load bibstyle here
     \RequirePackage[round,authoryear]{natbib} % CustomBib
 \fi

% changes the default name `Bibliography` -> `References'
\renewcommand{\bibname}{References}

% *****************************************************************************
% *************** Changing the Visual Style of Chapter Headings ***************
% Uncomment the section below. Requires titlesec package.

%\RequirePackage{titlesec}
%\newcommand{\PreContentTitleFormat}{\titleformat{\chapter}[display]{\scshape\Large}
%{\Large\filleft{\chaptertitlename} \Huge\thechapter}
%{1ex}{}
%[\vspace{1ex}\titlerule]}
%\newcommand{\ContentTitleFormat}{\titleformat{\chapter}[display]{\scshape\huge}
%{\Large\filleft{\chaptertitlename} \Huge\thechapter}{1ex}
%{\titlerule\vspace{1ex}\filright}
%[\vspace{1ex}\titlerule]}
%\newcommand{\PostContentTitleFormat}{\PreContentTitleFormat}
%\PreContentTitleFormat


% *****************************************************************************
% **************************** Custom Packages ********************************
% *****************************************************************************


% ************************* Algorithms and Pseudocode **************************

%\usepackage{algpseudocode} 


% ********************Captions and Hyperreferencing / URL **********************

% Captions: This makes captions of figures use a boldfaced small font. 
%\RequirePackage[small,bf]{caption}

\RequirePackage[labelsep=space,tableposition=top]{caption} 
\renewcommand{\figurename}{Fig.} %to support older versions of captions.sty

% ************************ Formatting / Footnote *******************************

%\usepackage[perpage]{footmisc} %Range of footnote options 
%IVAN: I added
\usepackage[export]{adjustbox}
%\usepackage{fontspec,lipsum} %i added for the fonsts
% ****************************** Line Numbers **********************************

%\RequirePackage{lineno}
%\linenumbers

% ************************** Graphics and figures *****************************
%\usepackage{mathgifg} %i added
%\usepackage{winfonts} for Georgia fonts
%\usepackage{mathgifg} for Georgia fonts
\usepackage[figuresright]{rotating}
\usepackage{commath} %for norm and absolute value
%\usepackage{wrapfig}
%\usepackage{float}
\usepackage{subfig} %note: subfig must be included after the `caption` package. 
\usepackage{nomencl}
% temporary
\usepackage[table]{xcolor}
%\usepackage[margin=1in]{geometry}
\usepackage{tabularx}
\usepackage{enumitem}
\setlist{nolistsep}
\definecolor{green}{HTML}{66FF66}
\definecolor{myGreen}{HTML}{009900}
%\renewcommand{\familydefault}{\sfdefault}
%\renewcommand{\arraystretch}{1.5}
% temporary
%for tables
\usepackage{booktabs} 
\usepackage{tabulary}
\usepackage{multirow}
\usepackage{geometry}

%%% for graph picture: Mehdi
% Graphics
\usepackage{graphicx}
\usepackage{tikz}
\usetikzlibrary{arrows}



% Other packages
\usepackage{enumitem}
\usepackage[boxed, vlined, linesnumbered]{algorithm2e}
\usepackage{authblk}

%%%%%%%%%%%%%%%%%%%%%%%%%%%%%%%%%
% blank page
\usepackage{afterpage}

\newcommand\blankpage{%
    \null
    \thispagestyle{empty}%
    \addtocounter{page}{-1}%
    \newpage}
    
%%%%%%%%%%%%%%%%%%%%%%%%%%%%%%%%%%%



% Colors for graph colorings
\definecolor{graphcol1}{rgb}{0.89, 0.10, 0.11}
\definecolor{graphcol2}{rgb}{0.22, 0.49, 0.72}
\definecolor{graphcol3}{rgb}{0.30, 0.69, 0.29}
\definecolor{graphcol4}{rgb}{0.60, 0.31, 0.64}

% Different edge types in graphs
\newlength{\edgelength}
\setlength{\edgelength}{2.5ex}
\newcommand{\grarright}{\mathbin{\tikz[baseline] \draw[->] (0pt, 0.7ex) -- (\edgelength, 0.7ex);}}
\newcommand{\grarleft}{\mathbin{\tikz[baseline] \draw[<-] (0pt, 0.7ex) -- (\edgelength, 0.7ex);}}
\newcommand{\grline}{\mathbin{\tikz[baseline] \draw[-] (0pt, 0.7ex) -- (\edgelength, 0.7ex);}}
\newcommand{\grdots}{\mathbin{\tikz[baseline] \draw[dotted, thick] (0pt, 0.7ex) -- (\edgelength, 0.7ex);}}



% Example graphs
\newlength{\exgredge}
\setlength{\exgredge}{9mm}


\newenvironment{fournodeex}{%
\begin{tikzpicture}[baseline=(v1.base)]
  \node[circle] (v1) at (0, 0) {$1$};
  \node [circle](v2) at (0, 1.5) {$2$};
  \node[circle] (v3) at (1.5, 1.5) {$3$};
  \node [circle](v4) at (1.5, 0) {$4$};
   \draw[black, very thick] (v1) circle (3mm);
     \draw[black, very thick] (v2) circle (3mm);
     \draw[black, very thick] (v3) circle (3mm);
     \draw[black, very thick] (v4) circle (3mm);
}
{\end{tikzpicture}}

% ********************************* Table **************************************

%\usepackage{longtable}
%\usepackage{multicol}
%\usepackage{multirow}
%\usepackage{tabularx}


% ***************************** Math and SI Units ******************************

\usepackage{amsfonts}
\usepackage{amsmath}
\allowdisplaybreaks % Ivan: I added this so he can brake formula on the next page
\usepackage{amssymb}
%\usepackage{siunitx} % use this package module for SI units

%mypackages 
\usepackage{color}
\usepackage{multirow}
\usepackage{amsfonts}
\usepackage{graphicx}
\usepackage{latexsym}
\usepackage{amsthm}
\usepackage{amsmath}
\usepackage{amssymb}
\usepackage{bm}
\usepackage{amsbsy}
%\usepackage{grffile}
\usepackage{ifpdf}
\usepackage{epstopdf}
\usepackage{epsfig,amssymb,amsfonts,verbatim}
\usepackage{fancyhdr,lastpage}
\usepackage{graphics}
\usepackage{color,soul}
\usepackage{setspace}
%\usepackage{subfiles}
%\usepackage{subfigure}


% ******************************************************************************
% ************************* User Defined Commands ******************************
% ******************************************************************************
\newtheorem{example}{Example}
\newtheorem{proposition}{Proposition}[chapter]
\newtheorem{theorem}{Theorem}[chapter]
\newtheorem{definition}{ Definition}[chapter]
\newtheorem{lemma}{Lemma}[chapter]
% ******************************************************************************
%http://tex.stackexchange.com/questions/154530/a-conditional-independence-symbol-that-looks-good-with-mid
\newcommand*{\bigCI}{%
  \mathrel{\text{%
    {\begin{tikzpicture}[baseline = {(0,0)}]%
        \draw[line width = 0.1ex] (0,0) -- (0,1.7ex);
        \draw[line width = 0.1ex] (0.3ex,0) -- (0.3ex,1.7ex);
        \draw[line width = 0.1ex] (-0.8ex,0) -- (1.1ex,0);
     \end{tikzpicture}%
    }%
  }}%
}
\newcommand{\eqdef}{\stackrel{\mathrm{def}}{=}}
\newcommand{\Rp}{\rR^p}
\newcommand{\cl}{\mathrm{ cl}}
\newcommand{\ybar}{\bar{y}}
\newcommand{\nei}{\mathrm{ne}}
%%%%%%%%%%%%%%%%%bolds%%%%%%%%%%%%%%%%%%%%%%%%%%
%%%%%%%%%%%%	GREEK bolds%%%%%%%%%%%%%%%%%%
%small
\newcommand{\alphab}{\bm{\alpha}}
\newcommand{\alphahatb}{\hat{\bm{\alpha}}}
\newcommand{\betab}{\bm{\beta}}
\newcommand{\xib}{\bm{\xi}}
\newcommand{\epsilonb}{\bm{\epsilon}}
\newcommand{\varepsilonb}{\bm{\varepsilon}}
\newcommand{\lambdab}{\bm{\lambda}}
\newcommand{\zetab}{\bm{\zeta}}
\newcommand{\etab}{\bm{\eta}}
\newcommand{\omegab}{\bm{\omega}}
\newcommand{\thetab}{\bm{\theta}}
\newcommand{\psib}{\bm{\psi}}
\newcommand{\phib}{\bm{\phi}}
\newcommand{\sigmab}{\bm{\sigma}}
\newcommand{\mub}{\bm{\mu}}
\newcommand{\muhatb}{\hat{\bm{\mu}}}
\newcommand{\thetabhat}{\hat{\bm\theta}}
\newcommand{\thetahatb}{\hat{\bm{\theta}}}
%capital
\newcommand{\Thetab}{\bm{\Theta}}
\newcommand{\Thetahatb}{\hat{\bm{\Theta}}}
\newcommand{\Thetabtilde}{\tilde{\bm{\Theta}}}
\newcommand{\Sigmab}{\bm{\Sigma}}
%%%%%%%%%%%%%%%%%bolds%%%%%%%%%%%%%%%%%%%%%%%%%%
%%%%%%%%%%%%	ROMAN bolds%%%%%%%%%%%%%%%%%%
%small
\newcommand{\zb}{\bm{z}}
\newcommand{\ab}{\bm{a}}
\newcommand{\cb}{\bm{c}}
\newcommand{\ub}{\bm{u}}
\newcommand{\xb}{\bm{x}}
\newcommand{\yb}{\bm{y}}
\newcommand{\fb}{\bm{f}}
\newcommand{\gb}{\mathbf{g}}
\newcommand{\ob}{\mathbf{0}}
\newcommand{\tb}{\bm{t}}
\newcommand{\xbdot}{\mathbf{\dot{x}}}
\newcommand{\cbhat}{\bm{\hat{c}}}
\newcommand{\xhatb}{\hat{\bm{x}}}
%capital
\newcommand{\Obi}{\bm{O}}
\newcommand{\Yb}{\bm{Y}}
\newcommand{\Ybmat}{\mathbf{Y}}
\newcommand{\Ab}{\mathbf{A}}
\newcommand{\Bb}{\mathbf{B}}
\newcommand{\Cb}{\mathbf{C}}
\newcommand{\Eb}{\mathbf{E}}
\newcommand{\Tb}{\mathbf{T}}
\newcommand{\Kb}{\mathbf{K}}
\newcommand{\Ib}{\mathbf{I}}
\newcommand{\Lb}{\mathbf{L}}
\newcommand{\Db}{\mathbf{D}}
\newcommand{\Fb}{\mathbf{F}}
\newcommand{\Pb}{\mathbf{P}}
\newcommand{\Gb}{\mathbf{G}}
\newcommand{\Mb}{\mathbf{M}}
\newcommand{\Rb}{\mathbf{R}}
\newcommand{\Qb}{\mathbf{Q}}
\newcommand{\Ob}{\mathbf{O}}
\newcommand{\Sb}{\mathbf{S}}
\newcommand{\Xb}{\mathbf{X}}
%%%%%%%%%%%%%%%%%%%%%%%%%%%%%%%%%%%%%%%%%%%%%%%%%%
%%%%%%   	TEXT 
\newcommand{\AIC}{\mathrm{ AIC}}
\newcommand{\BIC}{\mathrm{ BIC}}
\newcommand{\GIC}{\mathrm{ GIC}}
\newcommand{\EBIC}{\mathrm{ EBIC}}
\newcommand{\CV}{\mathrm{ CV}}
\newcommand{\GACV}{\mathrm{ GACV}}
\newcommand{\GCV}{\mathrm{ GCV}}
\newcommand{\KLCV}{\mathrm{ KLCV}}
\newcommand{\KLBIC}{\mathrm{ KLBIC}}
\newcommand{\KL}{\mathrm{ KL}}
\newcommand{\LOOCV}{\mathrm{ LOOCV}}
\newcommand{\Entropy}{\text{Entropy}}
\newcommand{\MCC}{\mathrm{ MCC}}
\newcommand{\Fone}{{rm F}_1}
\newcommand{\precision}{\mathrm{ precision}}
\newcommand{\recall}{\mathrm{ recall}}
\newcommand{\TP}{\mathrm{ TP}}
\newcommand{\TN}{\mathrm{ TN}}
\newcommand{\FP}{\mathrm{ FP}}
\newcommand{\FN}{\mathrm{ FN}}
\newcommand{\vect}{\mathrm{ vec}}
\newcommand{\tr}{\mathrm{ tr}}
\newcommand{\df}{\mathrm{ df}}
\newcommand{\bias}{\mathrm{ bias}}
\newcommand{\sign}{\mathrm{ sign}}
\newcommand{\diag}{\mathrm{ diag}}
\newcommand{\loglik}{\mathrm{ loglik}}
\newcommand{\argmax}{\mathrm{ argmax}}
\newcommand{\argmin}{\mathrm{ argmin}}
\newcommand{\E}{\mathrm{ E}}
\newcommand{\var}{\mathrm{ var}}
\newcommand{\cov}{\mathrm{ cov}}
\newcommand{\z}{\mathrm{ z}}
\newcommand{\x}{\mathrm{ x}}
\newcommand{\g}{\mathrm{ g}}
\newcommand{\F}{\mathrm{ F}}
\newcommand{\Y}{\mathrm{ Y}}
%%%%%%%%%%%%%%%%%%%%%%%%%%%%%%%%%%%%%%%%%%%%%%%%%%
\newcommand{\chat}{\hat{c}}
\newcommand{\lambdahat}{\hat{\lambda}}
\newcommand{\xidot}{\dot{\xi}}
\newcommand{\eps}{\varepsilon}
\newcommand{\concat}{\ensuremath{+\!\!\!\!+\,}} 
\newcommand{\Sigmahat}{\hat{\Sigma}}
%%%%%%%%%%%%%%%%%%%%%%%%%%%%%%%%%%%%%%%%%%%%%%%%%%
%%% 		hats, italics, caligraphic, tilde
\newcommand{\xhat}{\hat{x}}                                 
\newcommand{\Thetait}{{\it \Theta}}
\newcommand{\thetahat}{\hat{\theta}}
\newcommand{\sigmahat}{\hat{\sigma}}
\newcommand{\Ghatb}{\hat{\mathbf{G}}}
\newcommand{\muhat}{\hat{\mu}}
\newcommand{\Ltilde}{\tilde{L}}
\newcommand{\Stilde}{\tilde{S}}
\renewcommand{\hat}{\widehat}
\newcommand{\Hcal}{\mathcal{H}} 
\newcommand{\Dcal}{\mathcal{D}}     
\newcommand{\Acal}{\mathcal{A}}   
\newcommand{\Gcal}{\mathcal{G}} 
\newcommand{\Ncal}{\mathcal{N}} 
\newcommand{\Ocal}{\mathcal{O}} 
\newcommand{\Ycal}{{\mathcal Y}}
\renewcommand{\tilde}{\widetilde}
\newcommand{\rf}[1]{(\ref{eq:#1})}
\newcommand{\rR}{\mathbb R}
\newcommand{\rd}{\mathrm{d}}
\newcommand{\rP}{\mathrm{P}}
\newcommand{\rG}{\mathrm{G}}
\newcommand{\sd}{{\sf d}}
\newcommand{\sD}{{\sf D}}
\newcommand\dependent{\not\protect\mathpalette{\protect\independenT}{\perp}}
\newcommand\independent{\protect\mathpalette{\protect\independenT}{\perp}}
\def\independenT#1#2{\mathrel{\rlap{$#1#2$}\mkern2mu{#1#2}}}
%%%%%%%%%%%%%%
%RKHS notation
%%%%%%%%%%%%%%%%


\newcommand{\ceil}[1]{\lceil #1 \rceil}
\newcommand{\thh}{^\mathrm{th}}

\newcommand{\thetaof}[2]{\theta \langle #1;#2\rangle}
\newcommand{\Mpa}{M_\mathrm{P,A}}
\newcommand{\Ma}{M_\mathrm{A}}
\newcommand{\rjaccept}{\mathcal{A}}

\def\Skip{\par\bigskip\nobreak}
\def\dj{d\kern-0.4em\char"16\kern-0.1em}
\def\Dj{\mbox{\raise0.3ex\hbox{-}\kern-0.4em D}}
\def\bbbr{\mathrm{ I\!R}}
\def\bbbn{\mathrm{ I\!N}}
\def\bbbr{\mathrm{ I\!R}}
% ********************** TOC depth and numbering depth *************************

\setcounter{secnumdepth}{2}
\setcounter{tocdepth}{2}

% ******************************* Nomenclature *********************************

% To change the name of the Nomenclature section, uncomment the following line

\renewcommand\nomname{Symbols}


% ********************************* Appendix ***********************************

% The default value of both \appendixtocname and \appendixpagename is `Appendices'. These names can all be changed via: 

%\renewcommand{\appendixtocname}{List of appendices}

%\renewcommand{\appendixname}{Appndx}

